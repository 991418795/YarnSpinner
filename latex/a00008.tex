This is a quick-\/and-\/dirty document that shows how to add new functions so that they can be called from inside \hyperlink{a00026}{Yarn}.

First, get your {\ttfamily Dialogue} object\-: \begin{DoxyVerb}// some_implementation is the object that handles variable storage - 
// it's not important in this example
var dialogue = new Dialogue(some_implementation);
\end{DoxyVerb}


Next, register your new function. For example, let's make a function that takes 1 parameter, which is a number, and returns {\ttfamily true} if it's even\-:

\begin{DoxyVerb}dialogue.library.RegisterFunction ("is_even", 1, delegate(Value[] parameters) {
    return (int)parameters[0].AsNumber % 2 == 0;
});
\end{DoxyVerb}


When the function is called, the delegate you provide will be run.

Some notes\-:


\begin{DoxyItemize}
\item You don't have to return a value from your function.
\item The parameters passed to your function are of type \hyperlink{a00086}{Yarn.\-Value}. You can get their value as numbers, bools or strings by using the {\ttfamily As\-Number}, {\ttfamily As\-String} and {\ttfamily As\-Bool} properties.
\item You can only return values of the following types\-:
\begin{DoxyItemize}
\item String
\item Float
\item Double
\item Integer
\item \hyperlink{a00086}{Yarn.\-Value}
\end{DoxyItemize}
\item \hyperlink{a00026}{Yarn} Spinner will make sure that the correct number of parameters is passed to your method. If you specify {\ttfamily -\/1} as your parameter count, the function may have any number of parameters. 
\end{DoxyItemize}