\subsubsection*{Style Guide}

{\ttfamily Inline Code} contain short snippets of code for your project \begin{DoxyVerb}Code Blocks contain segments or chunks of code for your project
\end{DoxyVerb}


{\bfseries Bold indicates actions (select menu item, copying file, etc).}

$\ast$$\ast$$\ast$\-Bold italic text indicates emphasis.$\ast$$\ast$$\ast$

\begin{quotation}
Blockquotes contain essential information

\end{quotation}


\subsection*{Tutorial}

\begin{quotation}
$\ast$$\ast$$\ast$\-Note\-:$\ast$$\ast$$\ast$ This tutorial assumes that you know nothing about writing code, in particular how to use \href{http://www.unity3d.com}{\tt Unity} or the \hyperlink{a00050}{Yarn} Editor. You do not need to know how to use those tools to write \hyperlink{a00050}{Yarn} Dialogue. It is required that you know how to create a text file (.txt), which is different to a .doc, .docx, .rtf etc. document, in whatever text editor you are most comfortable using. Please refer to your friendly local computer expert if you are unsure how to do this. As you learn more about \hyperlink{a00050}{Yarn} Dialogue, you may find the \hyperlink{a00050}{Yarn} Editor useful as you progress with your authoring. It is a text editor that is designed to make \hyperlink{a00050}{Yarn} Dialogue easier to create.

\end{quotation}


\subsection*{Introducing \hyperlink{a00050}{Yarn} Dialogue}

\hyperlink{a00050}{Yarn} Dialogue is designed for Authors who have little or no programming knowledge. It makes no assumptions about how your game presents resultant text to the player, or about how the player chooses their responses. These tasks are left to the game production team, using programming and design tools such as Unity3\-D and C\#.

\subsubsection*{Simple Example Script}

We will create the essential \hyperlink{a00050}{Yarn} Dialogue for the \href{../Unity/Assets/YarnSpinner/Examples/DemoAssets/SimpleExampleScript.yarn.txt}{\tt Simple Example Dialogue}. We will start off with the text as presented by the ingame characters, add in choices which are essentially what the player says to the characters, and have the characters react to the choices.

Each section of text in \hyperlink{a00050}{Yarn} Dialogue is known as a $\ast$$\ast$$\ast$node$\ast$$\ast$$\ast$.

A $\ast$$\ast$$\ast$node$\ast$$\ast$$\ast$ consists of $\ast$$\ast$$\ast$header data$\ast$$\ast$$\ast$ and $\ast$$\ast$$\ast$body data$\ast$$\ast$$\ast$. The header data provides information about the body data, and the body data contains \hyperlink{a00050}{Yarn} Dialogue and options relating to the Dialogue.

In this example, our first node commences with two characters, A and B.

We need to give this node a title, and in this example we will call it $\ast$$\ast$$\ast$start$\ast$$\ast$$\ast$. There is no default title, it can be anything you like. ``` title\-: start ```

\begin{quotation}
Node titles are case sensitive. The titles $\ast$$\ast$$\ast$start$\ast$$\ast$$\ast$, $\ast$$\ast$$\ast$\-Start$\ast$$\ast$$\ast$, $\ast$$\ast$$\ast$\-S\-T\-A\-R\-T$\ast$$\ast$$\ast$ and $\ast$$\ast$$\ast$\-St\-Ar\-T$\ast$$\ast$$\ast$ are different.

\end{quotation}


\begin{quotation}
There is no default title. You will need to ensure that your game programmers know which node your game starts on, as it may not necessarily be the first node in your \hyperlink{a00050}{Yarn} Dialogue file.

\end{quotation}


The title tag is part of a collection of data known as $\ast$$\ast$$\ast$header data$\ast$$\ast$$\ast$ that can refer to position, colour and other information about the Dialogue. While the title tag is required, these other header data tags are optional and generally used later in game development when the Dialogue is attached to scenes and graphical characters.

To seperate the $\ast$$\ast$$\ast$header data$\ast$$\ast$$\ast$ from the $\ast$$\ast$$\ast$body data$\ast$$\ast$$\ast$ in the node, we then add in three {\ttfamily -\/} characters, known as a header delimiter. \subsection*{``` }

``` We can now add the starting text for our characters A and B. Their Dialogue is so\-: ``` A\-: Hey, I'm a character in a script! B\-: And I am too! You are talking to me!

``{\ttfamily  To indicate the end of the node, we add three}={\ttfamily characters. This is called a $\ast$$\ast$$\ast$node delimiter$\ast$$\ast$$\ast$ $<$h1$>$}``  ```

Your $\ast$$\ast$$\ast$node$\ast$$\ast$$\ast$ should now look like this ``` \subsection*{title\-: start }

A\-: Hey, I'm a character in a script! \section*{B\-: And I am too! You are talking to me! }

```

Congratulations, you've just written your very first piece of \hyperlink{a00050}{Yarn} Dialogue!

\subsubsection*{Adding Options and Replies}

\paragraph*{Options}

The next step we'll take is to add in some options to present to the player. We wish to present the player with the choice of two options. To do this, we use the {\ttfamily -\/$>$} marker as the first characters in our line of text to tell the system the text is an option.

We add this after the A and B character Dialogue, but before the ending tag. ``` -\/$>$ What's going on -\/$>$ Um ok ```

Our \hyperlink{a00050}{Yarn} Dialogue now reads as like this\-: ``` \subsection*{title\-: start }

A\-: Hey, I'm a character in a script! B\-: And I am too! You are talking to me! -\/$>$ What's going on \section*{-\/$>$ Um ok }

```

\paragraph*{Replies}

Options aren't really that useful without a $\ast$$\ast$$\ast$reply$\ast$$\ast$$\ast$. Replies are indented with four spaces. ``` A\-: Why this is a demo of the script system! B\-: And you're in it! ``` The reply is placed immediately after the option to which it is connected. We'll add the above replies to the first option, resulting in our \hyperlink{a00050}{Yarn} Dialogue now looking like this\-: ``` \subsection*{title\-: start }

A\-: Hey, I'm a character in a script! B\-: And I am too! You are talking to me! -\/$>$ What's going on A\-: Why this is a demo of the script system! B\-: And you're in it! \section*{-\/$>$ Um ok }

``` \begin{quotation}
Note\-: The order of the reply is important. In the above example, Character A replies, then Character B. We could easily reverse these lines if we wanted Character B to reply first.

\end{quotation}


Finishing up here, we'll end the Dialogue with a pleasantary.

``` A\-: How delightful! ``` We now have\-: ``` \subsection*{title\-: start }

A\-: Hey, I'm a character in a script! B\-: And I am too! You are talking to me! -\/$>$ What's going on A\-: Why this is a demo of the script system! B\-: And you're in it! -\/$>$ Um ok \section*{A\-: How delightful! }

```

\subsubsection*{Branching to New Nodes}

Additional nodes can be added to extend the story. These further nodes can introduce new characters, actions or provide different options and replies.

We will now present the player with options of selecting what to do next. ``` B\-: What would you prefer to do next? \mbox{[}\mbox{[}Leave$\vert$\-Leave\mbox{]}\mbox{]} \mbox{[}\mbox{[}Learn more$\vert$\-Learn\-More\mbox{]}\mbox{]} `{\ttfamily  The square double braces, `\mbox{[}\mbox{[}` and `\mbox{]}\mbox{]}`, indicate a branch to a different node. The vertical bar,}$\vert$``, separates the string to be presented to the player (the option), from the destination $\ast$$\ast$$\ast$node name$\ast$$\ast$$\ast$. Presentation of the new node will commence once the player has selected their chosen option text. Thus, the above code will present the player with two choices, 'Leave' and 'Learn More'. These options will take the player to the 'Leave' and 'Learn\-More' nodes respectively.

\begin{quotation}
Note\-: $\ast$$\ast$$\ast$\-Node names$\ast$$\ast$$\ast$ must be a single string, that is to say they cannot have spaces in them.

\end{quotation}


Our Dialogue now reads like this\-: ``` \subsection*{title\-: Start }

A\-: Hey, I'm a character in a script! B\-: And I am too! You are talking to me! -\/$>$ What's going on A\-: Why this is a demo of the script system! B\-: And you're in it! -\/$>$ Um ok A\-: How delightful! B\-: What would you prefer to do next? \mbox{[}\mbox{[}Leave$\vert$\-Leave\mbox{]}\mbox{]} \section*{\mbox{[}\mbox{[}Learn more$\vert$\-Learn\-More\mbox{]}\mbox{]} }

```

All we need to do now is add in two more nodes, the Leave and Learn\-More nodes. We'll add a little bit of conversation in to Characters A and B just to differentiate between the two nodes.

``` \subsection*{title\-: Leave }

A\-: Oh, goodbye! \section*{B\-: You'll be back soon! }

\subsection*{title\-: Learn\-More }

\section*{A\-: H\-A\-H\-A\-H\-A }

```

Our final \hyperlink{a00050}{Yarn} Dialogue file will read such\-: ``` \subsection*{title\-: Start }

A\-: Hey, I'm a character in a script! B\-: And I am too! You are talking to me! -\/$>$ What's going on A\-: Why this is a demo of the script system! B\-: And you're in it! -\/$>$ Um ok A\-: How delightful! B\-: What would you prefer to do next? \mbox{[}\mbox{[}Leave$\vert$\-Leave\mbox{]}\mbox{]} \section*{\mbox{[}\mbox{[}Learn more$\vert$\-Learn\-More\mbox{]}\mbox{]} }

\subsection*{title\-: Leave }

A\-: Oh, goodbye! \section*{B\-: You'll be back soon! }

\subsection*{title\-: Learn\-More }

\section*{A\-: H\-A\-H\-A\-H\-A }

```

\subsection*{Live example}

We have a fully parsed version of the above \hyperlink{a00050}{Yarn} Dialogue available in our Unity examples directory as \href{../Unity/Assets/YarnSpinner/Examples/DemoAssets/SimpleExampleScript.yarn.txt}{\tt Simple Example Script}. Please note this differs from the above \hyperlink{a00050}{Yarn} Dialogue in a couple of minor, yet important points\-:
\begin{DoxyItemize}
\item Extra header tags for colour and position of the Dialogue within the \hyperlink{a00050}{Yarn} Editor, and not for within the game environment. Text placement, colourising, and font selection is considered to be the domain of design and programming and thus outside the scope of Dialogue.
\item Translation tags to aide with localization These differences are generally created by programmers. It is envisaged that in the future they will be automatically generated by a \hyperlink{a00050}{Yarn} Dialogue editor. 
\end{DoxyItemize}