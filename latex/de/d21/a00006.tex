\begin{quotation}
$\ast$$\ast$$\ast$\-Important\-:$\ast$$\ast$$\ast$ This document only matters to you if you want to build \hyperlink{a00031}{Yarn} Spinner from source. In almost all cases, you'll be totally fine with \href{https://github.com/thesecretlab/YarnSpinner/releases}{\tt downloading a build}, and using that in your project.

\end{quotation}


To build \hyperlink{a00031}{Yarn} Spinner, you'll need Mono\-Develop. You can \href{http://www.monodevelop.com/download/}{\tt download Mono\-Develop for your O\-S from the Mono\-Develop site}.

Once it's downloaded, follow these steps\-:


\begin{DoxyItemize}
\item Open {\bfseries Yarn\-Spinner.\-sln}.
\item Open the {\bfseries Build menu}, and choose {\bfseries Build All}.
\item Open the {\bfseries Unity/\-Assets/\-Yarn Spinner} folder. You'll find a copy of {\bfseries Yarn\-Spinner.\-dll} there. You can now copy that D\-L\-L file wherever you need it.
\end{DoxyItemize}

\subsection*{Building Documentation}

\hyperlink{a00031}{Yarn} Spinner uses \href{https://www.stack.nl/~dimitri/doxygen}{\tt Doxygen} to generate \href{http://docbook.org/}{\tt Doc\-Book}, \href{https://en.wikipedia.org/wiki/HTML}{\tt H\-T\-M\-L}, \href{https://www.latex-project.org/help/documentation/}{\tt La\-Te\-X}, \href{https://en.wikipedia.org/wiki/Rich_Text_Format}{\tt R\-T\-F}, and \href{https://en.wikipedia.org/wiki/XML}{\tt X\-M\-L} documentation. \href{https://www.gnu.org/software/global/}{\tt G\-N\-U G\-L\-O\-B\-A\-L} is also used.

Basic steps to clean out existing documentation, generate new documentation and check the new documentation for generation errors. Note that some ocurrences of the word 'error' will be due to classes/methods etc. of \hyperlink{a00313}{Yarn\-Spinner} itself and not an actual error in the documentation.


\begin{DoxyItemize}
\item rm -\/fr Documentation/\{docbook,html,latex,rtf,xml\}
\item doxygen Documentation/\-Doxyfile $>$ doxyoutput.\-txt 2$>$\&1
\item grep -\/i error doxyoutput.\-txt
\end{DoxyItemize}

\href{https://daringfireball.net/projects/markdown/}{\tt Mark\-Down} documentation is available via conversion of the X\-M\-L output to Markdown using third party tools such as \href{http://pandoc.org}{\tt Pandoc} or \href{https://github.com/pferdinand/doxygen2md}{\tt doxygen2md} 