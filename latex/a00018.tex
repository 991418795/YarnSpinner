

{\bfseries \hyperlink{a00026}{Yarn} Spinner} is an interpreter for the \href{https://github.com/infiniteammoinc/Yarn}{\tt Yarn} language, written in C\#.

\begin{quotation}
{\bfseries New!} Join our \href{http://lab.to/narrativegamedev}{\tt narrative game development} Slack! {\bfseries New!} Continual integration \href{https://thesecretlab.github.io/YarnSpinner/html/}{\tt A\-P\-I documentation} now available

\end{quotation}


\hyperlink{a00026}{Yarn} is a language that's designed to make it {\itshape super easy} to create interactive dialogue for games. \hyperlink{a00026}{Yarn}'s very similar in style to \href{http://twinery.org}{\tt Twine}, so if you already know that, you'll be right at home! If you don't, that's cool -\/ \hyperlink{a00026}{Yarn}'s syntax is extremely minimal, and there's not much there to learn. The \hyperlink{a00026}{Yarn} language is used in a number of cool games, including \href{http://nightinthewoods.com}{\tt Night In The Woods} and \href{https://www.kickstarter.com/projects/foamsword/knights-and-bikes}{\tt Knights and Bikes}.

\begin{quotation}
$\ast$$\ast$$\ast$\-Important\-:$\ast$$\ast$$\ast$ \hyperlink{a00026}{Yarn} Spinner is still under development, and we haven't made our 1.\-0 release yet. It's {\itshape probably} fine to use right now, but there are a few bits and pieces that might change between now and first release.

\end{quotation}


\href{https://travis-ci.org/thesecretlab/YarnSpinner}{\tt !\mbox{[}Build Status\mbox{]}(https\-://travis-\/ci.\-org/thesecretlab/\-Yarn\-Spinner.\-svg?branch=master)}



\begin{quotation}
(Image from \char`\"{}\mbox{[}\-Night in the Woods\mbox{]}(http\-://nightinthewoods.\-com)\char`\"{} by Scott Benson, Bethany Hockenberry and Alec Holowka. Used with permission.)

\end{quotation}


\hyperlink{a00026}{Yarn} Spinner is designed to be easy to add to Unity games, but it's also intended for use in other contexts as well.

\subsection*{Quick Start}


\begin{DoxyEnumerate}
\item Download the \href{https://github.com/InfiniteAmmoInc/Yarn}{\tt Yarn editor}, so that you can create and edit \hyperlink{a00026}{Yarn} files.
\end{DoxyEnumerate}
\begin{DoxyEnumerate}
\item Create a new project in \href{https://unity3d.com/get-unity}{\tt Unity}.
\item Download the \href{https://github.com/thesecretlab/YarnSpinner/releases}{\tt latest release's Unity package}.
\item Import the package.
\item Go to the {\ttfamily \hyperlink{a00026}{Yarn} Spinner/\-Examples/} folder in Unity, and open one of the demo scenes.
\item Run the game, and poke around!
\end{DoxyEnumerate}

\subsection*{What To Do Next}


\begin{DoxyItemize}
\item \href{https://github.com/thesecretlab/YarnSpinner/releases}{\tt Download Yarn Spinner.}
\item \hyperlink{a00006}{Learn how to build Yarn Spinner from source.}
\item \href{https://github.com/infiniteammoinc/Yarn}{\tt Learn more about Yarn, and the Yarn editor.}
\item \hyperlink{a00014}{Learn more about writing dialogue in Yarn.}
\item \hyperlink{a00012}{Learn about using Yarn Spinner in your Unity game.}
\end{DoxyItemize}

\href{http://shop.oreilly.com/product/0636920055105.do}{\tt }

We also produced a training video on integrating and using the basics of \hyperlink{a00026}{Yarn} Spinner in Unity games, \href{http://shop.oreilly.com/product/0636920055105.do}{\tt available commercially from O'Reilly Media} (and \href{https://www.safaribooksonline.com/library/view/creating-narrative-games/9781491969830/}{\tt also on Safari}).

\subsection*{License}

\hyperlink{a00026}{Yarn} Spinner is available under the \hyperlink{a00016}{M\-I\-T License}. This means that you can use it in any commercial or noncommercial project. The only requirement is that you need to include attribution in your game's docs. A credit would be very, very nice, too, but isn't required.


\begin{DoxyItemize}
\item \href{https://tldrlegal.com/license/mit-license}{\tt Learn more about what this license lets you do}.
\end{DoxyItemize}

\subsection*{Made by Secret Lab!}

\hyperlink{a00026}{Yarn} Spinner was originally created by \href{http://secretlab.com.au}{\tt Secret Lab}, an Australian game dev studio! \href{https://twitter.com/thesecretlab}{\tt Come say hi}! You can visit the \href{http://www.secretlab.com.au/yarnspinner}{\tt Yarn Spinner page} on the Secret Lab website for a little more info, and to donate to \hyperlink{a00026}{Yarn} Spinner open source development at Secret Lab.

The awesome \href{Documentation/YarnSpinnerLogo.png}{\tt logo} was made by the excellent \href{https://twitter.com/RexSmeal}{\tt Rex Smeal}, and is under the \href{http://creativecommons.org/licenses/by-sa/4.0/}{\tt Creative Commons Attribution-\/\-Share\-Alike 4.\-0 International License}.

\subsection*{Help Us Make \hyperlink{a00026}{Yarn} Spinner!}

\hyperlink{a00026}{Yarn} Spinner needs your help to be as awesome as it can be! You don't have to be a coder to help out -\/ we'd love to have your help in improving our \hyperlink{a00012}{documentation}, in spreading the word, and in finding bugs.


\begin{DoxyItemize}
\item Our \href{https://github.com/thesecretlab/YarnSpinner/issues}{\tt issues page} contains a list of things we'd love your help in improving.
\item Hop into our I\-R\-C channel, which is \href{https://webchat.freenode.net/?channels=yarnspinner}{\tt \#yarnspinner on Freenode}, to chat to the team, lend a hand, or ask questions.
\item {\bfseries New!} Join our \href{http://lab.to/narrativegamedev}{\tt narrative game development} Slack!
\item {\bfseries New!} Continual integration \href{https://thesecretlab.github.io/YarnSpinner/html/}{\tt A\-P\-I documentation} now available
\end{DoxyItemize}

If you want to contribute to \hyperlink{a00026}{Yarn} Spinner (!!), \hyperlink{a00002}{go read our contributor's guide!} 