This document is intended to act as a comprehensive and concise reference for \hyperlink{a00053}{Yarn} syntax and structure, for use by programmers and content creators. It assumes a working knowledge of modern programming/scripting languages. For a more thorough explanation of \hyperlink{a00053}{Yarn} usage, see the \hyperlink{a00213}{General Usage Guide}

\subsection*{Table of Contents}


\begin{DoxyItemize}
\item \href{#nodes}{\tt Nodes}
\item \href{#links-between-nodes}{\tt Links Between Nodes}
\begin{DoxyItemize}
\item \href{#menu-syntax}{\tt Menu Syntax}
\item \href{#option-syntax}{\tt Option Syntax}
\end{DoxyItemize}
\item \href{#variables---conditionals}{\tt Variables \& Conditionals}
\begin{DoxyItemize}
\item \href{#declaring-and-setting-variables}{\tt Declaring and Setting Variables}
\item \href{#variable-types}{\tt Variable Types}
\item \href{#if-else-statements}{\tt If/\-Else Statements}
\item \href{#operator-synonyms}{\tt Operator Synonyms}
\end{DoxyItemize}
\item \href{#commands-and-functions}{\tt Commands and Functions}
\end{DoxyItemize}

\subsubsection*{Nodes}

Nodes act as containers for \hyperlink{a00053}{Yarn} script, and must have unique titles. The script in the body of a node is processed line by line.

A node's header contains its metadata -\/ by default, \hyperlink{a00053}{Yarn} only uses the {\ttfamily title} field, but can be extended to use arbitrary fields.

``` title\-: Example\-Node\-Name \subsection*{tags\-: foo, bar }

\hyperlink{a00053}{Yarn} content goes here. This is the second line. 

 ```

A script file can contain multiple nodes. In this case, nodes are delineated using three equals ({\ttfamily =}) characters.

Additionally, \hyperlink{a00053}{Yarn} can check if a node has been visited by calling {\ttfamily visited(\char`\"{}\-Node\-Name\char`\"{})} in an {\ttfamily if} statement (i.\-e. {\ttfamily $<$$<$if visited(\char`\"{}\-Node\-Name\char`\"{}) == true$>$$>$}).

\subsubsection*{Links Between Nodes}

Nodes link to other nodes through {\itshape options}. An option is composed of a label (optional) and a node name separated by a vertical bar ({\ttfamily $\vert$}), like so\-:

``` \mbox{[}\mbox{[}A Link To A Node$\vert$\-Node1\mbox{]}\mbox{]} ```

If a node link with no label is provided ({\ttfamily \mbox{[}\mbox{[}Node1\mbox{]}\mbox{]}}), \hyperlink{a00053}{Yarn} will automatically navigate to the linked node.

\paragraph*{Menu Syntax}

Shortcut options allow for small branches in \hyperlink{a00053}{Yarn} scripts without requiring extra nodes. Shortcut option sets allow for an arbitrary number of sub-\/branches, but it's recommended that users stick to as few as possible for the sake of script readability.

``` Mae\-: What did you say to her? -\/$>$ Nothing. Mae\-: Oh, man. Maybe you should have. -\/$>$ That she was a jerk. Mae\-: Hah! I bet that pissed her off. Mae\-: How'd she react? -\/$>$ She didn't. Mae\-: Booooo. That's boring. -\/$>$ Furiously. Mae\-: That's what I like to hear! Mae\-: Anyway, I'd better get going. ```

Additionally, shortcut options can utilize conditional logic, commands and functions (detailed below), and can include standard node links. If a condition is attached to a shortcut option, the option will only appear to the reader if the condition passes\-:

``` Bob\-: What would you like? -\/$>$ A burger. $<$$<$if \$money $>$ 5$>$$>$ Bob\-: Nice. Enjoy! \mbox{[}\mbox{[}Ate\-A\-Burger\mbox{]}\mbox{]} -\/$>$ A soda. $<$$<$if \$money $>$ 2$>$$>$ Bob\-: Yum! \mbox{[}\mbox{[}Drank\-A\-Soda\mbox{]}\mbox{]} -\/$>$ Nothing. Bob\-: Okay. Bob\-: Thanks for coming! ```

\paragraph*{Option Syntax}

Multiple labeled node links on consecutive lines will be parsed as a menu. Example\-:

``` \mbox{[}\mbox{[}Option 1$\vert$\-Node1\mbox{]}\mbox{]} \mbox{[}\mbox{[}Option the Second$\vert$\-Node2\mbox{]}\mbox{]} \mbox{[}\mbox{[}Third Option$\vert$\-Node3\mbox{]}\mbox{]} ```

\subsection*{Variables \& Conditionals}

\paragraph*{Declaring and Setting Variables}

``` $<$$<$set \$\-Example\-Variable to 1$>$$>$ ```

This statement serves to set a variable's value. No declarative statement is required; setting a variable's value brings it into existence.

Variable names must start with a {\ttfamily \$} character.

\paragraph*{Variable Types}

There are four different types of variable in \hyperlink{a00053}{Yarn}\-: strings, floating-\/point numbers, \href{http://eesemi.com/boolean.htm}{\tt booleans}, and {\ttfamily null}.

\hyperlink{a00053}{Yarn} will automatically convert between types. For example\-:

``` $<$$<$if \char`\"{}hi\char`\"{} == \char`\"{}hi\char`\"{}$>$$>$ The two strings are the same! $<$$<$endif$>$$>$

$<$$<$if 1+1+\char`\"{}hi\char`\"{} == \char`\"{}2hi\char`\"{}$>$$>$ Strings get joined together with other values! $<$$<$endif$>$$>$ ```

\paragraph*{If/\-Else Statements}

\hyperlink{a00053}{Yarn} supports standard if/else/elseif statements.

``` $<$$<$if \char`\"{}hi\char`\"{} == \char`\"{}hi\char`\"{}$>$$>$ The two strings are the same! $<$$<$endif$>$$>$ ```

``` $<$$<$if \$variable == 1$>$$>$ Success! $<$$<$elseif \$variable == \char`\"{}hello\char`\"{}$>$$>$ Success...? $<$$<$else$>$$>$ No success. \-:( $<$$<$endif$>$$>$ ```

\paragraph*{Operator Synonyms}

\subparagraph*{Assignment}

\begin{TabularC}{2}
\hline
\rowcolor{lightgray}\PBS\centering {\bf word }&\PBS\centering {\bf symbol  }\\\cline{1-2}
\PBS\centering {\ttfamily to} &\PBS\centering {\ttfamily =} \\\cline{1-2}
\end{TabularC}
\subparagraph*{Comparison}

\begin{TabularC}{2}
\hline
\rowcolor{lightgray}\PBS\centering {\bf word }&\PBS\centering {\bf symbol  }\\\cline{1-2}
\PBS\centering {\ttfamily and} &\PBS\centering {\ttfamily \&} \\\cline{1-2}
\PBS\centering {\ttfamily le} &\PBS\centering {\ttfamily $<$} \\\cline{1-2}
\PBS\centering {\ttfamily gt} &\PBS\centering {\ttfamily $>$} \\\cline{1-2}
\end{TabularC}
$\vert$ {\ttfamily or} $\vert$ {\ttfamily $\vert$$\vert$} $\vert$ $\vert$ {\ttfamily leq} $\vert$ {\ttfamily $<$=} $\vert$ $\vert$ {\ttfamily geq} $\vert$ {\ttfamily $>$=} $\vert$ $\vert$ {\ttfamily eq} $\vert$ {\ttfamily ==} $\vert$ $\vert$ {\ttfamily is} $\vert$ {\ttfamily ==} $\vert$ $\vert$ {\ttfamily neq} $\vert$ {\ttfamily !=} $\vert$ $\vert$ {\ttfamily not} $\vert$ {\ttfamily !} $\vert$

\subsection*{Commands and Functions}

By default, \hyperlink{a00053}{Yarn} Spinner includes a {\ttfamily visited()} function, used to check whether a node has been entered. ``` $<$$<$if visited(\char`\"{}\-Go\-To\-City\char`\"{})$>$$>$ We have gone to the city before! $<$$<$endif$>$$>$ ``` Additional functions and commands can be added at run-\/time. See ../\-Yarn\-Spinner-\/\-Programming/\-Extending.md \char`\"{}\char`\"{}Extending \hyperlink{a00053}{Yarn} Spinner\char`\"{}\char`\"{} for more info. 