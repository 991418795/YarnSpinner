\subsubsection*{Style Guide}

{\ttfamily Inline Code} contain short snippets of code for your project \begin{DoxyVerb}Code Blocks contain segments or chunks of code for your project
\end{DoxyVerb}


{\bfseries Bold indicate actions (Select menu item, copying file, etc.)}

$\ast$$\ast$$\ast$\-Bold italic text indicates emphasis$\ast$$\ast$$\ast$

\begin{quotation}
Blockquotes contain essential information

\end{quotation}


\subsection*{Tutorial}

\begin{quotation}
$\ast$$\ast$$\ast$\-Note\-:$\ast$$\ast$$\ast$ This tutorial assumes that you know a little bit about \href{http://www.unity3d.com}{\tt Unity}. In particular, it is helpful that you know how to get around the Unity editor, how to work with game objects, and how to write scripts in C\#. If you don't know these things, please refer to \href{http://unity3d.com/learn}{\tt Unity's documentation}.

\end{quotation}


\subsection*{\hyperlink{a00040}{Yarn} Spinner Quick Start}

Here's how to quickly jump in to \hyperlink{a00040}{Yarn} Spinner if you're already familiar with Unity.


\begin{DoxyItemize}
\item {\bfseries Download} \href{https://github.com/thesecretlab/YarnSpinner/releases}{\tt the Yarn\-Spinner package} into your Unity project.
\item {\bfseries Import} the package into your project.
\item Inside the \hyperlink{a00175}{Yarn\-Spinner} folder, {\bfseries open the {\ttfamily Examples/\-Yarn Spinner Basic Example} scene.}
\item {\bfseries Start the game.}
\item {\bfseries Play through the dialogue.}
\end{DoxyItemize}

Once you've played with it, open the Example Script file in the \href{http://github.com/infiniteammoinc/Yarn}{\tt Yarn Editor}, (it's in the {\ttfamily Examples/\-Demo Assets} folder), and make some changes to the script. Once you've done that, take a look at how {\ttfamily \hyperlink{a00126}{Code/\-Dialogue\-Runner.\-cs}}, {\ttfamily Examples/\-Demo \hyperlink{a00130}{Scripts/\-Example\-Dialogue\-U\-I.\-cs}} and {\ttfamily Examples/\-Demo \hyperlink{a00131}{Scripts/\-Example\-Variable\-Storage.\-cs}} work. You can also \hyperlink{a00120}{add your own functions to Yarn}. 